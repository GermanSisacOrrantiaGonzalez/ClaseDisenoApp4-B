\documentclass{article}
\usepackage{graphicx} % Required for inserting images
\usepackage{makeidx}
\usepackage{float} % Necesario para usar la opción [H]
\usepackage{cancel}



\title{Propuesta de proyecto.}
\author{Integrantes: \\
Bautista Ordoñez Brian Angel. \\
Garcia Valenzuela Ernesto. \\
Islas Reyes Luis Ivan. \\
Ruelas Gónzalez Carlos Alexis. \\
Orrantia González German Sisac. \\ \\
Líder de equipo: \\
Bautista Ordoñez Brian Angel.}

\date{February 2024} 
\makeindex
\begin{document}
\begin{figure}
    \centering
    \includegraphics[width=0.5\linewidth]{utt.png}
    \label{fig:enter-label}
\end{figure}


\maketitle


  \index{1 Introducción } 
  \index{2 Descripción del Proyecto} 
  \index{3 Nuevas Implementaciones}
  \index{4 Foto del cronograma}
  \index{5 Requisitos para la página de nóminas.}


\renewcommand{\indexname}{Índice} % Cambiar el nombre del índice
\printindex

\newpage
\section{Introducción.}
En este archivo se podrá encontrar la descripción del proyecto, así como el problema que viene a resolver, acompañado de las nuevas tecnologías que se utilizaran para actualizar este, tomando en cuenta los requerimientos establecidos anteriormente y siguiendo el cronograma de actividades establecido para la realización del proyecto. \\

\section{Descripción del proyecto.}
TFT es una aplicación web dedicada a la gestión automatizada del cálculo de nóminas con enfoque a pequeñas y medianas empresas. TFT permite a la empresa crear perfiles (tipos de empleado), crear prestaciones, otorgar prestaciones, registrar la entrada y salida del personal, personalizar los sueldos, añadir a sus empleados, modificar la información, etc. todo esto para generar de manera automatizada la nómina para sus empleados a quienes la aplicación les permite visualizar su nómina más reciente, un historial de nóminas y ciertas estadísticas. El modelo de negocios para TFT es un modelo de suscripción donde la empresa puede escoger entre 2 paquetes (mensual/anual) según sus necesidades.
La problemática que viene a solucionar TFT son los altos costes y difícil acceso que tienen las pequeñas y medianas empresas para acceder a tecnologías que facilitan la gestión de sus nóminas. Con TFT estas empresas podrán optimizar esta área de su empresa de una manera fácil, eficiente y económica.

\section{Nuevas Implementaciones.}
Para la materia de IoT:
\begin{itemize}
    \item Se planea hacer un scanner de reconocimiento de huella digital.
\end{itemize}
\begin{itemize}
    \item Se planea hacer un scanner de Código QR para el inicio de sesión.
\end{itemize}

Para la materia de Aplicaciones web:
\begin{itemize}
    \item Se planea la implementación de Apis para el uso de la página.
\end{itemize}

Y por último para la materia de Diseño de apps:
\begin{itemize}
    \item Se planea hacer una aplicación móvil.
\end{itemize}

\newpage

\section{Foto del cronograma.}
   \begin{figure}[H] % La opción [H] fuerza la posición
    \centering
    \includegraphics[width=1.2\linewidth]{si.jpeg}
    \label{fig:enter-label}
\end{figure}

   
\newpage
\section{Requisitos para la página.}
%ESTE ES LA TABLA DEL REQUISITO 1
\begin{center}
\begin{tabular}{|l|l|}
\hline
Número de requisito & RN1 \\
\hline
Nombre de requisito & Contrato del servicio \\
\hline
Tipo & \cancel{Requisito} | Restricción \\
\hline
Fuente del requisito & Página encargada de mostrar los diferentes paquetes del servicio \\
\hline
Prioridad del requisito & \cancel{Alta/Esencial} \\
                       & Media/Deseado \\
                       & Baja/Opcional \\
\hline
Nivel de dificultad & 1 \\
\hline
\end{tabular}
\end{center}

%ESTE ES LA TABLA DEL REQUISITO 2
\begin{center}
\begin{tabular}{|l|l|}
\hline
Número de requisito & RN2 \\
\hline
Nombre de requisito & Portal de pagos. \\
\hline
Tipo & \cancel{Requisito} | Restricción \\
\hline
Fuente del requisito & \shortstack {Apartado encargado de la recopilación de la información para \\ completar el pago de servicio (simulación).} \\
\hline
Prioridad del requisito & \cancel{Alta/Esencial} \\
                       & Media/Deseado \\
                       & Baja/Opcional \\
\hline
Nivel de dificultad & 1 \\
\hline
\end{tabular}
\end{center}

%ESTE ES LA TABLA DEL REQUISITO 3
\begin{center}
\begin{tabular}{|l|l|}
\hline
Número de requisito & RN3 \\
\hline
Nombre de requisito & Registro de cuenta. \\
\hline
Tipo & \cancel{Requisito} | Restricción \\
\hline
Fuente del requisito & Página para registrar cuenta empresarial. \\
\hline
Prioridad del requisito & \cancel{Alta/Esencial} \\
                       & Media/Deseado \\
                       & Baja/Opcional \\
\hline
Nivel de dificultad & 1 \\
\hline
\end{tabular}
\end{center}

%ESTE ES LA TABLA DEL REQUISITO 4
\begin{center}
\begin{tabular}{|l|l|}
\hline
Número de requisito & RN4 \\
\hline
Nombre de requisito & Inicio de sesión empresarial. \\
\hline
Tipo & \cancel{Requisito} | Restricción \\
\hline
Fuente del requisito & Apartado para el inicio de sesión empresarial. \\
\hline
Prioridad del requisito & \cancel{Alta/Esencial} \\
                       & Media/Deseado \\
                       & Baja/Opcional \\
\hline
Nivel de dificultad & 1 \\
\hline
\end{tabular}
\end{center}

%ESTE ES LA TABLA DEL REQUISITO 5
\begin{center}
\begin{tabular}{|l|l|}
\hline
Número de requisito & RN5 \\
\hline
Nombre de requisito & Inicio de sesión de administradores. \\
\hline
Tipo & \cancel{Requisito} | Restricción \\
\hline
Fuente del requisito & Apartado para el inicio de sesión de administradores dueños de la página web. \\
\hline
Prioridad del requisito & \cancel{Alta/Esencial} \\
                       & Media/Deseado \\
                       & Baja/Opcional \\
\hline
Nivel de dificultad & 1 \\
\hline
\end{tabular}
\end{center}

%ESTE ES LA TABLA DEL REQUISITO 6
\begin{center}
\begin{tabular}{|l|l|}
\hline
Número de requisito & RN6 \\
\hline
Nombre de requisito & Inicio de sesión de empleados. \\
\hline
Tipo & \cancel{Requisito} | Restricción \\
\hline
Fuente del requisito & Apartado para el inicio de sesión empleado(Los registra la empresa). \\
\hline
Prioridad del requisito & \cancel{Alta/Esencial} \\
                       & Media/Deseado \\
                       & Baja/Opcional \\
\hline
Nivel de dificultad & 1 \\
\hline
\end{tabular}
\end{center}

%ESTE ES LA TABLA DEL REQUISITO 7
\begin{center}
\begin{tabular}{|l|l|}
\hline
Número de requisito & RN7 \\
\hline
Nombre de requisito & Cálculo de ingresos.\\
\hline
Tipo & \cancel{Requisito}  Restricción \\
\hline
Fuente del requisito & \shortstack{Programación de la lógica necesaria para el cálculo \\
de salario basado en las necesidades de la empresa deberá ser la suma de: \\
- Sueldo Ordinario. \\ -Séptimo día. \\ -Prima domonical. \\ -Subsidio para el empleo \\ -Prestaciones.}
   \\
\hline
Prioridad del requisito & \cancel{Alta/Esencial} \\
                       & Media/Deseado \\
                       & Baja/Opcional \\
\hline
Nivel de dificultad & 2 \\
\hline
\end{tabular}
\end{center}


%ESTE ES LA TABLA DEL REQUISITO 8

\begin{center}
\begin{tabular}{|l|l|}
\hline
Número de requisito & RN8 \\
\hline
Nombre de requisito & Cálculo de deducciones.\\
\hline
Tipo & \cancel{Requisito}  Restricción \\
\hline
Fuente del requisito & \shortstack{Programación de la lógica necesaria para el cálculo de las deducciones salariales,\\ deberá ser la suma de: \\
- ISR. \\ -IMSS. \\ -Prestaciones.}
   \\
\hline
Prioridad del requisito & \cancel{Alta/Esencial} \\
                       & Media/Deseado \\
                       & Baja/Opcional \\
\hline
Nivel de dificultad & 2 \\
\hline
\end{tabular}
\end{center}

%ESTE ES LA TABLA DEL REQUISITO 9 TYMO SEXO

\begin{center}
\begin{tabular}{|l|l|}
\hline
Número de requisito & \parbox{10cm}{RN9} \\
\hline
Nombre de requisito & \parbox{10cm}{Creación y edición de perfiles.} \\
\hline
Tipo & \cancel{Requisito}  Restricción \\
\hline
Fuente del requisito & \parbox{10cm}{En este apartado la empresa será capaz de crear un perfil dedicado a un tipo de empleado así mismo podrá establecer el pago de este tipo de empleados y sus prestaciones.} \\
\hline
Prioridad del requisito & \cancel{Alta/Esencial} \\
                       & Media/Deseado \\
                       & Baja/Opcional\\
\hline
Nivel de dificultad & \parbox{10cm}{3} \\
\hline
\end{tabular}
\end{center}


%ESTE ES LA TABLA DEL REQUISITO 10 TYMO SEXO

\begin{center}
\begin{tabular}{|l|l|}
\hline
Número de requisito & \parbox{10cm}{RN10} \\
\hline
Nombre de requisito & \parbox{10cm}{Registro de empleados} \\
\hline
Tipo & \cancel{Requisito}  Restricción  \\
\hline
Fuente del requisito & \parbox{10cm}{En este apartado la empresa será capaz de registrar a todos sus empleados y asignarlos a un perfil anteriormente creado por medio de un correo que la empresa se encarga de conseguir.} \\
\hline
Prioridad del requisito & \cancel{Alta/Esencial} \\
                       & Media/Deseado \\
                       & Baja/Opcional \\
\hline
Nivel de dificultad & \parbox{10cm}{1} \\
\hline
\end{tabular}
\end{center}



%ESTE ES LA TABLA DEL REQUISITO 11
\begin{center}
\begin{tabular}{|l|l|}
\hline
Número de requisito & RN11 \\
\hline
Nombre de requisito & Estadisticas de la empresa.\\
\hline
Tipo & \cancel{Requisito}  Restricción \\
\hline
Fuente del requisito & \shortstack{En este apartado se encontrarán las siguientes estadísticas de la empresa: \\
- Costo total de Nómina. \\ -Distribución de costes de nómina.}
   \\
\hline
Prioridad del requisito & \cancel{Alta/Esencial} \\
                       & Media/Deseado \\
                       & Baja/Opcional \\
\hline
Nivel de dificultad & 2 \\
\hline
\end{tabular}
\end{center}

%ESTE ES LA TABLA DEL REQUISITO 12 TYMO SEXO
\begin{adjustwidth}{-3cm}{-3cm}
\begin{center}
\begin{tabular}{|l|l|}
\hline
Número de requisito & \parbox{10cm}{RN12} \\
\hline
Nombre de requisito & \parbox{10cm}{Historial de pagos y deducciones} \\
\hline
Tipo & \cancel{Requisito}  Restricción  \\
\hline
Fuente del requisito & \parbox{10cm}{Registros detallados de los pagos realizados a cada empleado, incluyendo detalles de salarios, bonificaciones y deducciones.} \\
\hline
Prioridad del requisito & \cancel{Alta/Esencial} \\
                       & Media/Deseado \\
                       & Baja/Opcional \\
\hline
Nivel de dificultad & \parbox{10cm}{3} \\
\hline
\end{tabular}
\end{center}
\end{adjustwidth}


%ESTE ES LA TABLA DEL REQUISITO 13 TYMO SEXO
\begin{adjustwidth}{-3cm}{-3cm}
\begin{center}
\begin{tabular}{|l|l|}
\hline
Número de requisito & \parbox{10cm}{RN13} \\
\hline
Nombre de requisito & \parbox{10cm}{Configuración de cuenta} \\
\hline
Tipo & \cancel{Requisito}  Restricción \\
\hline
Fuente del requisito & \parbox{10cm}{En este apartado la empresa podrá modificar la información de su empresa.} \\
\hline
Prioridad del requisito & \cancel{Alta/Esencial} \\
                       & Media/Deseado \\
                       & Baja/Opcional \\
\hline
Nivel de dificultad & \parbox{10cm}{1} \\
\hline
\end{tabular}
\end{center}
\end{adjustwidth}

%ESTE ES LA TABLA DEL REQUISITO 14 TYMO SEXO
\begin{adjustwidth}{-3cm}{-3cm}
\begin{center}
\begin{tabular}{|l|l|}
\hline
Número de requisito & \parbox{10cm}{RN14} \\
\hline
Nombre de requisito & \parbox{10cm}{Nómina más reciente.} \\
\hline
Tipo & \cancel{Requisito}  Restricción\\
\hline
Fuente del requisito & \parbox{10cm}{En este apartado podrá ver el recibo de su nómina más reciente.} \\
\hline
Prioridad del requisito & \cancel{Alta/Esencial} \\
                       & Media/Deseado \\
                       & Baja/Opcional \\
\hline
Nivel de dificultad & \parbox{10cm}{1} \\
\hline
\end{tabular}
\end{center}
\end{adjustwidth}

%ESTE ES LA TABLA DEL REQUISITO 15 TYMO SEXO
\begin{adjustwidth}{-3cm}{-3cm}
\begin{center}
\begin{tabular}{|l|l|}
\hline
Número de requisito & \parbox{10cm}{RN15} \\
\hline
Nombre de requisito & \parbox{10cm}{Historial de nóminas.} \\
\hline
Tipo & \cancel{Requisito}  Restricción \\
\hline
Fuente del requisito & \parbox{10cm}{En este apartado podrá acceder a todos los recibos de nóminas que ha recibido.} \\
\hline
Prioridad del requisito & \cancel{Alta/Esencial} \\
                       & Media/Deseado \\
                       & Baja/Opcional \\
\hline
Nivel de dificultad & \parbox{10cm}{3} \\
\hline
\end{tabular}
\end{center}
\end{adjustwidth}

%ESTE ES LA TABLA DEL REQUISITO 16
\begin{center}
\begin{tabular}{|l|l|}
\hline
Número de requisito & RN16 \\
\hline
Nombre de requisito & Estadísticas de ingresos.\\
\hline
Tipo & \cancel{Requisito}  Restricción \\
\hline
Fuente del requisito & \shortstack{En este apartado el empleado podrá ver las siguientes \\ estadísticas de sus ingresos: \\
- Ingreso anual. \\ - Ingresos exentos. \\ - Ingresos acumulables. \\ -Subsidio para el empleo \\ - Impuesto retenido.}
   \\
\hline
Prioridad del requisito & \cancel{Alta/Esencial} \\
                       & Media/Deseado \\
                       & Baja/Opcional \\
\hline
Nivel de dificultad & 2 \\
\hline
\end{tabular}
\end{center}

%ESTE ES LA TABLA DEL REQUISITO 17 TYMO SEXO

\begin{center}
\begin{tabular}{|l|l|}
\hline
Número de requisito & \parbox{10cm}{RN17} \\
\hline
Nombre de requisito & \parbox{10cm}{Información de empleado} \\
\hline
Tipo & \cancel{Requisito}  Restricción  \\
\hline
Fuente del requisito & \parbox{10cm}{En este apartado podrá ver su información.} \\
\hline
Prioridad del requisito & \cancel{Alta/Esencial} \\
                       & Media/Deseado \\
                       & Baja/Opcional\\
\hline
Nivel de dificultad & \parbox{10cm}{1} \\
\hline
\end{tabular}
\end{center}


%ESTE ES LA TABLA DEL REQUISITO 18 TYMO SEXO

\begin{center}
\begin{tabular}{|l|l|}
\hline
Número de requisito & \parbox{10cm}{RN18} \\
\hline
Nombre de requisito & \parbox{10cm}{Generación de recibo de pago de nómina.} \\
\hline
Tipo & \cancel{Requisito}  Restricción \\
\hline
Fuente del requisito & \parbox{10cm}{En este apartado el empleado podrá generar un recibo de sus pagos de nómina.} \\
\hline
Prioridad del requisito & \cancel{Alta/Esencial} \\
                       & Media/Deseado \\
                       & Baja/Opcional\\
\hline
Nivel de dificultad & \parbox{10cm}{1} \\
\hline
\end{tabular}
\end{center}


%ESTE ES LA TABLA DEL REQUISITO 19 TYMO SEXO

\begin{center}
\begin{tabular}{|l|l|}
\hline
Número de requisito & \parbox{10cm}{RN19} \\
\hline
Nombre de requisito & \parbox{10cm}{Gestión de empleados} \\
\hline
Tipo & \cancel{Requisito}  Restricción  \\
\hline
Fuente del requisito & \parbox{10cm}{Apartado dedicado al manejo de empleados, en este apartado se podrá modificar la información de este mismo, así como darlo de baja.} \\
\hline
Prioridad del requisito & \cancel{Alta/Esencial} \\
                       & Media/Deseado \\
                       & Baja/Opcional\\
\hline
Nivel de dificultad & \parbox{10cm}{2} \\
\hline
\end{tabular}
\end{center}



\end{document} 