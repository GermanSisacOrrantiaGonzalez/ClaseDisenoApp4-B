\documentclass{article}
\usepackage{graphicx} % Required for inserting images

\title {Assignment No2 - App Design}
\author {Sisac Orrantia}
\date {January 3, 2024}

\begin{document}

\maketitle{}
\section{Native and Non-Native Applications}

\paragraph{Native Applications:}
Refer to those designed specifically to operate on a particular platform, taking advantage of the characteristics and functionalities of that particular operating system. These applications are developed using programming languages and tools native to the target platform, enabling optimal performance and full integration with the device.

\subparagraph{Characteristics:}
\paragraph{1. Optimized Performance:}
Native applications benefit from superior performance as they are directly optimized for the operating system and hardware of the device, making the most of its capabilities.

\paragraph{2. Superior User Experience:}
Being specifically designed for a particular platform, native applications offer a smooth and consistent user experience, adapted to the design guidelines and functionalities of the operating system.

\paragraph{Examples of Native Applications:}
\paragraph{1- Instagram (iOS and Android):}
Instagram is a prominent example of a native application, providing a consistent and optimized experience on iOS and Android platforms. It leverages the specific characteristics of each operating system to provide functions such as the camera, push notifications, and smooth scrolling.

\paragraph{2- Apple Music (iOS):}
This application is native to iOS and is designed to maximize the capabilities of Apple devices. It offers seamless integration with the Apple ecosystem, including specific functions such as Siri integration and efficient hardware usage.

\paragraph{Non-Native Applications (Hybrid or Web):}
These applications are developed using standard web technologies such as HTML, CSS, and JavaScript, then wrapped in a native container to enable distribution and execution on different platforms. Often, they cannot fully leverage the specific functionalities of the device.

\subparagraph{Characteristics:}
\paragraph{1. Cross-Platform Development:}
Non-native applications offer the advantage of being developed once and running on multiple platforms, saving time and resources.

\paragraph{2. Lower Performance and Limited User Experience:}
Although faster to develop, these applications often have slightly lower performance and a less optimized user experience compared to native ones due to limitations in accessing device functionalities.

\paragraph{Examples of Non-Native Applications:}
\paragraph{1- Twitter (web version and hybrid application):}
The web version of Twitter and its hybrid application offer a similar experience across different platforms but may lack some specific device functionalities available in the native application.

\paragraph{2- LinkedIn (hybrid application):}
LinkedIn also offers a hybrid application that operates on multiple platforms. While providing access to the social network, it may have limitations compared to its native counterpart in terms of performance and some specific device functionalities.

\end{document}
